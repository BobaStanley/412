% Options for packages loaded elsewhere
\PassOptionsToPackage{unicode}{hyperref}
\PassOptionsToPackage{hyphens}{url}
%
\documentclass[
  man,floatsintext]{apa6}
\usepackage{amsmath,amssymb}
\usepackage{iftex}
\ifPDFTeX
  \usepackage[T1]{fontenc}
  \usepackage[utf8]{inputenc}
  \usepackage{textcomp} % provide euro and other symbols
\else % if luatex or xetex
  \usepackage{unicode-math} % this also loads fontspec
  \defaultfontfeatures{Scale=MatchLowercase}
  \defaultfontfeatures[\rmfamily]{Ligatures=TeX,Scale=1}
\fi
\usepackage{lmodern}
\ifPDFTeX\else
  % xetex/luatex font selection
\fi
% Use upquote if available, for straight quotes in verbatim environments
\IfFileExists{upquote.sty}{\usepackage{upquote}}{}
\IfFileExists{microtype.sty}{% use microtype if available
  \usepackage[]{microtype}
  \UseMicrotypeSet[protrusion]{basicmath} % disable protrusion for tt fonts
}{}
\makeatletter
\@ifundefined{KOMAClassName}{% if non-KOMA class
  \IfFileExists{parskip.sty}{%
    \usepackage{parskip}
  }{% else
    \setlength{\parindent}{0pt}
    \setlength{\parskip}{6pt plus 2pt minus 1pt}}
}{% if KOMA class
  \KOMAoptions{parskip=half}}
\makeatother
\usepackage{xcolor}
\usepackage{graphicx}
\makeatletter
\def\maxwidth{\ifdim\Gin@nat@width>\linewidth\linewidth\else\Gin@nat@width\fi}
\def\maxheight{\ifdim\Gin@nat@height>\textheight\textheight\else\Gin@nat@height\fi}
\makeatother
% Scale images if necessary, so that they will not overflow the page
% margins by default, and it is still possible to overwrite the defaults
% using explicit options in \includegraphics[width, height, ...]{}
\setkeys{Gin}{width=\maxwidth,height=\maxheight,keepaspectratio}
% Set default figure placement to htbp
\makeatletter
\def\fps@figure{htbp}
\makeatother
\setlength{\emergencystretch}{3em} % prevent overfull lines
\providecommand{\tightlist}{%
  \setlength{\itemsep}{0pt}\setlength{\parskip}{0pt}}
\setcounter{secnumdepth}{-\maxdimen} % remove section numbering
% Make \paragraph and \subparagraph free-standing
\ifx\paragraph\undefined\else
  \let\oldparagraph\paragraph
  \renewcommand{\paragraph}[1]{\oldparagraph{#1}\mbox{}}
\fi
\ifx\subparagraph\undefined\else
  \let\oldsubparagraph\subparagraph
  \renewcommand{\subparagraph}[1]{\oldsubparagraph{#1}\mbox{}}
\fi
\newlength{\cslhangindent}
\setlength{\cslhangindent}{1.5em}
\newlength{\csllabelwidth}
\setlength{\csllabelwidth}{3em}
\newlength{\cslentryspacingunit} % times entry-spacing
\setlength{\cslentryspacingunit}{\parskip}
\newenvironment{CSLReferences}[2] % #1 hanging-ident, #2 entry spacing
 {% don't indent paragraphs
  \setlength{\parindent}{0pt}
  % turn on hanging indent if param 1 is 1
  \ifodd #1
  \let\oldpar\par
  \def\par{\hangindent=\cslhangindent\oldpar}
  \fi
  % set entry spacing
  \setlength{\parskip}{#2\cslentryspacingunit}
 }%
 {}
\usepackage{calc}
\newcommand{\CSLBlock}[1]{#1\hfill\break}
\newcommand{\CSLLeftMargin}[1]{\parbox[t]{\csllabelwidth}{#1}}
\newcommand{\CSLRightInline}[1]{\parbox[t]{\linewidth - \csllabelwidth}{#1}\break}
\newcommand{\CSLIndent}[1]{\hspace{\cslhangindent}#1}
\ifLuaTeX
\usepackage[bidi=basic]{babel}
\else
\usepackage[bidi=default]{babel}
\fi
\babelprovide[main,import]{english}
% get rid of language-specific shorthands (see #6817):
\let\LanguageShortHands\languageshorthands
\def\languageshorthands#1{}
% Manuscript styling
\usepackage{upgreek}
\captionsetup{font=singlespacing,justification=justified}

% Table formatting
\usepackage{longtable}
\usepackage{lscape}
% \usepackage[counterclockwise]{rotating}   % Landscape page setup for large tables
\usepackage{multirow}		% Table styling
\usepackage{tabularx}		% Control Column width
\usepackage[flushleft]{threeparttable}	% Allows for three part tables with a specified notes section
\usepackage{threeparttablex}            % Lets threeparttable work with longtable

% Create new environments so endfloat can handle them
% \newenvironment{ltable}
%   {\begin{landscape}\centering\begin{threeparttable}}
%   {\end{threeparttable}\end{landscape}}
\newenvironment{lltable}{\begin{landscape}\centering\begin{ThreePartTable}}{\end{ThreePartTable}\end{landscape}}

% Enables adjusting longtable caption width to table width
% Solution found at http://golatex.de/longtable-mit-caption-so-breit-wie-die-tabelle-t15767.html
\makeatletter
\newcommand\LastLTentrywidth{1em}
\newlength\longtablewidth
\setlength{\longtablewidth}{1in}
\newcommand{\getlongtablewidth}{\begingroup \ifcsname LT@\roman{LT@tables}\endcsname \global\longtablewidth=0pt \renewcommand{\LT@entry}[2]{\global\advance\longtablewidth by ##2\relax\gdef\LastLTentrywidth{##2}}\@nameuse{LT@\roman{LT@tables}} \fi \endgroup}

% \setlength{\parindent}{0.5in}
% \setlength{\parskip}{0pt plus 0pt minus 0pt}

% Overwrite redefinition of paragraph and subparagraph by the default LaTeX template
% See https://github.com/crsh/papaja/issues/292
\makeatletter
\renewcommand{\paragraph}{\@startsection{paragraph}{4}{\parindent}%
  {0\baselineskip \@plus 0.2ex \@minus 0.2ex}%
  {-1em}%
  {\normalfont\normalsize\bfseries\itshape\typesectitle}}

\renewcommand{\subparagraph}[1]{\@startsection{subparagraph}{5}{1em}%
  {0\baselineskip \@plus 0.2ex \@minus 0.2ex}%
  {-\z@\relax}%
  {\normalfont\normalsize\itshape\hspace{\parindent}{#1}\textit{\addperi}}{\relax}}
\makeatother

\makeatletter
\usepackage{etoolbox}
\patchcmd{\maketitle}
  {\section{\normalfont\normalsize\abstractname}}
  {\section*{\normalfont\normalsize\abstractname}}
  {}{\typeout{Failed to patch abstract.}}
\patchcmd{\maketitle}
  {\section{\protect\normalfont{\@title}}}
  {\section*{\protect\normalfont{\@title}}}
  {}{\typeout{Failed to patch title.}}
\makeatother

\usepackage{xpatch}
\makeatletter
\xapptocmd\appendix
  {\xapptocmd\section
    {\addcontentsline{toc}{section}{\appendixname\ifoneappendix\else~\theappendix\fi\\: #1}}
    {}{\InnerPatchFailed}%
  }
{}{\PatchFailed}
\keywords{Best, GOAT, Fighter, Performance, UFC\newline\indent Word count: 160}
\usepackage{lineno}

\linenumbers
\usepackage{csquotes}
\ifLuaTeX
  \usepackage{selnolig}  % disable illegal ligatures
\fi
\IfFileExists{bookmark.sty}{\usepackage{bookmark}}{\usepackage{hyperref}}
\IfFileExists{xurl.sty}{\usepackage{xurl}}{} % add URL line breaks if available
\urlstyle{same}
\hypersetup{
  pdftitle={Into the UFC: Best Fighter, Striker, Grappler, and Entertainer},
  pdfauthor={Stanley Go1},
  pdflang={en-EN},
  pdfkeywords={Best, GOAT, Fighter, Performance, UFC},
  hidelinks,
  pdfcreator={LaTeX via pandoc}}

\title{Into the UFC: Best Fighter, Striker, Grappler, and Entertainer}
\author{Stanley Go\textsuperscript{1}}
\date{}


\shorttitle{UFC Stats}

\authornote{

Add complete departmental affiliations for each author here. Each new line herein must be indented, like this line.

Enter author note here.

The authors made the following contributions. Stanley Go: Conceptualization, Writing - Original Draft Preparation, Writing - Review \& Editing.

Correspondence concerning this article should be addressed to Stanley Go, Postal address. E-mail: \href{mailto:smg421@scarletmail.rutgers.edu}{\nolinkurl{smg421@scarletmail.rutgers.edu}}

}

\affiliation{\vspace{0.5cm}\textsuperscript{1} Rutgers University}

\abstract{%
This project delves into an in-depth analysis of UFC fighter statistics to identify and recognize excellence in various performance categories. The primary objective is to pinpoint the best fighters in distinct areas such as striking, grappling, knockout ability, and overall entertainment value. The study further aims to determine the greatest of all time (GOAT) by evaluating fighters based on their win-loss ratios. Leveraging a comprehensive dataset encompassing fighter metrics, the analysis employs key indicators such as significant strikes rate, takedown success rate, and total knockdowns. The anticipated results include detailed insights into the best striker, grappler, KOer, and entertainer, contributing to a comprehensive understanding of individual strengths within the competitive realm of UFC. Additionally, the study seeks to establish the GOAT by considering the historical performance of fighters with a minimum threshold of 10 matches. Through meticulous analysis and visualization, this project aims to offer a nuanced perspective on the unparalleled skills and accomplishments of UFC fighters across diverse categories.
}



\begin{document}
\maketitle

\#Introduction:

\begin{itemize}
\tightlist
\item
  Background:

  \begin{itemize}
  \tightlist
  \item
    Brief overview of the UFC and its significance in the world of mixed martial arts (MMA).
  \item
    Growing interest in understanding and analyzing fighter performance.
  \end{itemize}
\item
  Research Focus:

  \begin{itemize}
  \tightlist
  \item
    Exploration of UFC fighter statistics to recognize excellence in various performance categories.
  \end{itemize}
\item
  Big Question:

  \begin{itemize}
  \tightlist
  \item
    The Ultimate Fighting Championship (UFC) hosts a wide range of fighters with diverse skill sets, including striking, grappling, and entertainment value. The objective of this project is to systematically analyze UFC fighter data to determine who excels in each category, identify the best overall fighters, and predict fight outcomes and potential earnings. We will aim our analysis to provide insights into fighter performance, audience appeal, and financial success within the UFC.
  \end{itemize}
\item
  Objectives:

  \begin{itemize}
  \tightlist
  \item
    Identification of the best fighters in specific areas:

    \begin{itemize}
    \tightlist
    \item
      Best Striker
    \item
      Best Grappler
    \item
      Best KOer
    \item
      Best Entertainer
    \end{itemize}
  \end{itemize}
\item
  Key Questions:

  \begin{itemize}
  \tightlist
  \item
    Research questions guiding the analysis:

    \begin{itemize}
    \tightlist
    \item
      Who are the best strikers, grapplers, and KOers in UFC?
    \item
      What factors contribute to a fighter's entertainment value?
    \item
      Who is considered the greatest of all time in UFC?
    \end{itemize}
  \end{itemize}
\item
  Data Source:

  \begin{itemize}
  \tightlist
  \item
    Utilization of the UFC Stats dataset with a comprehensive set of fighter metrics.
  \end{itemize}
\item
  Methodology:

  \begin{itemize}
  \tightlist
  \item
    Planned analysis includes calculating key metrics such as significant strikes rate, takedown success rate, and total knockdowns.
  \end{itemize}
\item
  Anticipated Results:

  \begin{itemize}
  \tightlist
  \item
    Expected outcomes involve insights into the best performers in each category, contributing to a nuanced understanding of individual strengths.
  \end{itemize}
\item
  Significance:

  \begin{itemize}
  \tightlist
  \item
    Emphasis on the significance of recognizing and appreciating excellence within the competitive realm of UFC.
  \end{itemize}
\item
  Overview of the Document:

  \begin{itemize}
  \tightlist
  \item
    Brief mention of the subsequent sections, including the methodology, results, and discussion.
  \end{itemize}
\end{itemize}

\hypertarget{methods}{%
\section{Methods}\label{methods}}

\hypertarget{data-collection}{%
\subsection{Data Collection}\label{data-collection}}

\begin{itemize}
\tightlist
\item
  \textbf{Data Source:}

  \begin{itemize}
  \tightlist
  \item
    Utilize the UFC Stats dataset, containing comprehensive fighter metrics, including but not limited to knockdowns, significant strikes, takedowns, and fight outcomes.
  \end{itemize}
\end{itemize}

\hypertarget{analysis-approach}{%
\subsection{Analysis Approach}\label{analysis-approach}}

\begin{enumerate}
\def\labelenumi{\arabic{enumi}.}
\tightlist
\item
  \textbf{Identifying Best Striker:}

  \begin{itemize}
  \tightlist
  \item
    Calculate the significant strikes rate for each fighter:
    \[ \text{Significant Strikes Rate} = \frac{\text{Total Significant Strikes Landed}}{\text{Total Significant Strikes Attempted}} \]
  \item
    Determine the fighter with the highest significant strikes rate as the best striker.
  \end{itemize}
\item
  \textbf{Identifying Best Grappler:}

  \begin{itemize}
  \tightlist
  \item
    Compute the takedown success rate for each fighter:
    \[ \text{Takedown Success Rate} = \frac{\text{Total Successful Takedowns}}{\text{Total Takedown Attempts}} \]
  \item
    Identify the fighter with the highest takedown success rate as the best grappler.
  \end{itemize}
\item
  \textbf{Identifying Best KOer:}

  \begin{itemize}
  \tightlist
  \item
    Sum the total knockdowns for each fighter.
  \item
    Determine the fighter with the highest total knockdowns as the best KOer.
  \end{itemize}
\item
  \textbf{Identifying Best Entertainer:}

  \begin{itemize}
  \tightlist
  \item
    Develop a composite metric considering both significant strikes and successful takedowns.
  \item
    Determine the fighter with the highest composite metric as the best entertainer.
  \end{itemize}
\item
  \textbf{Determining Greatest of All Time (GOAT):}

  \begin{itemize}
  \tightlist
  \item
    Filter fighters with a minimum threshold of 10 matches.
  \item
    Calculate the win-loss ratio for each fighter:
    \[ \text{Win-Loss Ratio} = \frac{\text{Total Wins}}{\text{Total Wins + Total Losses}} \]
  \item
    Identify the fighter with the highest win-loss ratio as the GOAT.
  \end{itemize}
\end{enumerate}

\hypertarget{visualization}{%
\subsection{Visualization}\label{visualization}}

\begin{itemize}
\tightlist
\item
  \textbf{Figure Creation:}

  \begin{itemize}
  \tightlist
  \item
    Develop bar plots for each category (Best Striker, Best Grappler, Best KOer, Best Entertainer, and GOAT) to visually represent the analysis results.
  \end{itemize}
\end{itemize}

\hypertarget{limitations}{%
\subsection{Limitations}\label{limitations}}

\begin{itemize}
\tightlist
\item
  \textbf{Limitations of the Analysis:}

  \begin{itemize}
  \tightlist
  \item
    Acknowledge potential limitations such as data accuracy, sample size variations, and the subjectivity of composite metrics.
  \end{itemize}
\end{itemize}

\hypertarget{ethical-considerations}{%
\subsection{Ethical Considerations}\label{ethical-considerations}}

\begin{itemize}
\tightlist
\item
  \textbf{Ethical Considerations:}

  \begin{itemize}
  \tightlist
  \item
    Ensure the privacy and consent of fighters in the analysis.
  \item
    Avoid biased interpretations and represent results objectively.
  \end{itemize}
\end{itemize}

\hypertarget{participants}{%
\subsection{Participants}\label{participants}}

\hypertarget{material}{%
\subsection{Material}\label{material}}

\hypertarget{procedure}{%
\subsection{Procedure}\label{procedure}}

\hypertarget{data-analysis}{%
\subsection{Data analysis}\label{data-analysis}}

We used R (Version 4.3.1; R Core Team, 2023) and the R-packages \emph{papaja} (Version 0.1.2; Aust \& Barth, 2023), and \emph{tinylabels} (Version 0.2.4; Barth, 2023) for all our analyses.

\hypertarget{results}{%
\section{Results}\label{results}}

\hypertarget{discussion}{%
\section{Discussion}\label{discussion}}

\newpage

\hypertarget{references}{%
\section{References}\label{references}}

\hypertarget{refs}{}
\begin{CSLReferences}{1}{0}
\leavevmode\vadjust pre{\hypertarget{ref-R-papaja}{}}%
Aust, F., \& Barth, M. (2023). \emph{{papaja}: {Prepare} reproducible {APA} journal articles with {R Markdown}}. Retrieved from \url{https://github.com/crsh/papaja}

\leavevmode\vadjust pre{\hypertarget{ref-R-tinylabels}{}}%
Barth, M. (2023). \emph{{tinylabels}: Lightweight variable labels}. Retrieved from \url{https://cran.r-project.org/package=tinylabels}

\leavevmode\vadjust pre{\hypertarget{ref-R-base}{}}%
R Core Team. (2023). \emph{R: A language and environment for statistical computing}. Vienna, Austria: R Foundation for Statistical Computing. Retrieved from \url{https://www.R-project.org/}

\end{CSLReferences}


\end{document}
